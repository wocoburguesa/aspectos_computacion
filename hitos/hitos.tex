\documentclass[a4paper,10pt]{article}
\usepackage[utf8]{inputenc}
\usepackage[spanish]{babel}
\usepackage{enumitem}
\usepackage{graphicx}

\title{Hitos matemáticos en la historia de la computación}
\author{\textbf{Marco Flores Nuñez}}

\begin{document}
\maketitle

\section{Álgebra booleana}

El álgebra booleana fue introducida en 1854 por el matemático inglés George Boole en su libro \textit{An Investigation of the Laws of Thought}. 

George Boole falleció en 1864, pero sus ideas han vivido y evolucionado mucho más allá de lo que su creador pudo prever. Para aclarar, a pesar del nombre, el álgebra booleana como la conocemos hoy no es un aporte exclusivo del fallecido inglés, muchos académicos han contribuido a su ya mencionada evolución.

En la década de 1930, por ejemplo, Claude Shannon (cabe mencionar, entre los logros del señor Shannon, ser conocido como "El padre de la teoría de la información") observó que se podían aplicar las ideas de Boole en un contexto de conmutación de circuitos, utilizando el concepto de puertas lógicas e introdujo el "álgebra de conmutaciones". Hoy en día la necesidad de considerar distintos tipos de álgebras booleanas es mínima, por lo tanto "álgebra de conmutaciones" y "álgebra booleana" se usan refiriéndose a la misma cosa. Todo esto está detallado en la tesis de maestría de Shannon, que algunos aclaman como la tesis de maestría más importante de todos los tiempos.

Al afirmar en el párrafo anterior que existen distintos tipos de álgebras booleanas 

\end{document}
